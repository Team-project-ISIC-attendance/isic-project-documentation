\chapter{Ciele riešenia}

Cieľom tímového projektu je navrhnúť a realizovať funkčný prototyp systému na evidenciu účasti študentov na vyučovacích hodinách s využitím RFID technológie a jeho čiastočnou integráciou do existujúceho akademického informačného systému. Riešenie je koncipované ako moderný, automatizovaný a technicky realizovateľný systém, ktorý reflektuje reálne potreby akademického prostredia a zároveň zohľadňuje technologické a organizačné obmedzenia univerzitného prostredia.

Na obrázku \ref{o:konceptualna-schema} je znázornená konceptuálna schéma navrhovaného riešenia, ktorá ilustruje základnú myšlienku systému evidencie účasti a vzťahy medzi jeho hlavnými aktérmi. Schéma poskytuje prehľad o toku informácií od študenta prostredníctvom identifikačného média až po centrálny informačný systém, pričom zdôrazňuje automatizovaný charakter navrhovaného riešenia.

\begin{figure}[!ht]
    \centering
    \includegraphics[width=0.8\textwidth]{figures/overall-system-concept}
    \caption{Konceptuálna schéma systému evidencie účasti študentov}
    \label{o:konceptualna-schema}
\end{figure}


Projekt sa zameriava na vytvorenie uceleného prototypu, ktorý demonštruje možnosť nahradenia manuálnych foriem evidencie dochádzky spoľahlivejším a transparentnejším spôsobom. Dôraz je kladený na preukázanie praktickej využiteľnosti navrhovaného riešenia, nie na jeho plnohodnotné produkčné nasadenie. Výsledný systém má slúžiť ako dôkaz konceptu, ktorý môže byť v budúcnosti ďalej rozširovaný a prispôsobovaný konkrétnym podmienkam fakúlt alebo univerzít.

\section{Hlavný cieľ riešenia}

Hlavným cieľom riešenia je návrh a implementácia prototypu systému, ktorý umožní automatizovanú evidenciu účasti študentov na vyučovaní prostredníctvom RFID identifikácie a následné spracovanie týchto údajov v centrálnej softvérovej aplikácii. Systém má zabezpečiť spoľahlivý zber údajov o dochádzke, ich ukladanie do databázy a prípravu dát na ďalšie spracovanie alebo export do akademického informačného systému bez potreby manuálneho zásahu zo strany pedagóga.

Riešenie má zároveň overiť vhodnosť RFID technológie pre tento typ použitia v akademickom prostredí, najmä z pohľadu presnosti identifikácie, rýchlosti spracovania a praktickej použiteľnosti počas reálneho vyučovania.

\section{Čiastkové ciele riešenia}

Jedným z čiastkových cieľov je návrh hardvérovej časti systému vo forme samostatného RFID zariadenia, ktoré bude schopné identifikovať študentské karty, spracovať získané údaje a bezpečne ich odoslať na server prostredníctvom bezdrôtovej komunikácie. Dôraz je kladený na jednoduchosť zapojenia a konfigurovateľnosť zariadenia tak, aby bolo možné jeho použitie v rôznych učebniach bez zložitej technickej prípravy.

Ďalším cieľom je návrh a realizácia softvérovej časti systému vo forme webovej aplikácie, ktorá bude slúžiť ako centrálny bod pre správu údajov o účasti. Táto aplikácia má zabezpečiť správu používateľov, spracovanie údajov prijatých z RFID zariadení a ich ukladanie do databázovej štruktúry navrhnutej s ohľadom na akademický kontext.

Samostatným cieľom riešenia je vytvorenie mechanizmu pre export a import údajov o účasti v štandardizovaných formátoch, ktoré umožnia následnú integráciu s akademickým informačným systémom. Tento cieľ reflektuje potrebu interoperability a minimalizácie manuálnej administratívnej práce zo strany vyučujúcich a administrátorov.

\section{Didaktické a praktické ciele projektu}

Projekt má zároveň vzdelávací charakter a jeho cieľom je aplikovať teoretické poznatky získané počas štúdia na riešenie reálneho problému. Riešenie poskytuje priestor na overenie schopnosti tímovej spolupráce, návrhu systémovej architektúry a integrácie hardvérových a softvérových komponentov do jedného funkčného celku.

Dôležitým cieľom je tiež získanie praktických skúseností s návrhom prototypových riešení v oblasti internetu vecí a webových informačných systémov, pričom dôraz je kladený na systematický prístup, dokumentovanie procesu vývoja a hodnotenie dosiahnutých výsledkov z hľadiska ich ďalšej rozšíriteľnosti a využiteľnosti v akademickej praxi.
