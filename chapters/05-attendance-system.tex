\chapter{Ako funguje ISIC Attendance System}

Táto kapitola popisuje architektúru a tok údajov v prototypovom systéme ISIC Attendance System. Zameriava sa na prepojenie fyzickej identifikácie študenta pomocou ISIC karty s digitálnym záznamom dochádzky, ktorý je následne sprístupnený vo webovej aplikácii. Text je písaný s dôrazom na jasné oddelenie komponentov a ich zodpovedností.

\section{Hlavné komponenty systému}

Riešenie je navrhnuté modulárne a pozostáva z hardvérového zariadenia v učebni, serverovej časti, webového rozhrania a databázy. Komponenty sú prepojené tak, aby bol zber udalostí spoľahlivý, prenos efektívny a ukladanie údajov konzistentné.

\subsection{Hardware: zariadenie v učebni}
V každej učebni je umiestnené kompaktné zariadenie s RFID/NFC čítačkou určené na evidenciu prítomnosti. Prototyp je postavený na mikrokontroléri \textit{ESP8266 / ESP-12F} a NFC module \textit{PN532}. Zariadenie sa pripája do lokálnej Wi-Fi siete a odosiela udalosti na server prostredníctvom protokolu \textit{MQTT}, ktorý je vhodný pre scenáre s väčším počtom zariadení a častými udalosťami.

\subsubsection{Zoznam súčiastok a napájanie}
Zariadenie je navrhnuté ako jednoduchý IoT uzol s NFC čítačkou a bezdrôtovou komunikáciou. Z pohľadu hardvéru prototyp obsahuje najmä:
\begin{itemize}
  \item modul \textit{ESP-12F (ESP8266)} ako hlavná riadiaca jednotka,
  \item modul \textit{PN532} na čítanie NFC/RFID kariet (ISIC),
  \item napájanie \textbf{3,3~V} (v prototypovej verzii uvažované ako batériový zdroj 3,3~V),
  \item indikačné prvky (LED + bzučiak/piezo) pre spätnú väzbu používateľovi,
  \item mechanické časti (kryt, konektory, prepojovacie vodiče).
\end{itemize}

Keďže \textit{ESP-12F} pracuje výhradne s logikou 3,3~V, napájanie 3,3~V umožňuje pripojiť modul \textit{PN532} bez potreby úrovňových prevodníkov a zjednodušuje zapojenie.

\subsubsection{Prepojenie PN532 a ESP-12F}
Komunikácia medzi \textit{ESP-12F} a čítačkou \textit{PN532} je v prototypovej implementácii realizovaná cez zbernicu \textit{SPI} (HSPI na ESP8266). Tento spôsob umožňuje stabilnú komunikáciu a zároveň zodpovedá použitej implementácii firmvéru.

\begin{table}[h!]
  \centering
  \caption{Navrhované prepojenie PN532 s ESP-12F (SPI)}
  \label{tab:pn532-esp12f-wiring}
  \begin{tabular}{ll}
    \toprule
    PN532 & ESP-12F (ESP8266) \\
    \midrule
    VCC (3,3~V) & 3,3~V \\
    GND & GND \\
    SCK & GPIO14 \\
    MISO & GPIO12 \\
    MOSI & GPIO13 \\
    SS/CS & GPIO5 \\
    IRQ & GPIO4 \\
    RSTO (voliteľné) & reset (alebo voľný GPIO) \\
    \bottomrule
  \end{tabular}
\end{table}

V prototypovej fáze je pre funkčnosť čítania potrebné prepojenie napájania, zeme a SPI signálov (SCK/MOSI/MISO/CS). Pin \textit{IRQ} je v aktuálnej implementácii použitý pre signalizáciu udalosti z čítačky (predvolene GPIO4) a \textit{RSTO} je voliteľný (najmä pre hardvérový reset alebo pri pokročilejších režimoch šetrenia energie).

\subsubsection{Prepojenie LED a bzučiaka}
Spätná väzba pre používateľa je realizovaná cez \texttt{FeedbackService}, pričom konkrétne piny sú konfigurovateľné (\texttt{FeedbackConfig}). V prototypovej verzii bez kolízie so SPI je použité nasledujúce zapojenie:
\begin{itemize}
  \item \textbf{LED}: GPIO2 (D4) -- často je k dispozícii vstavaná LED (aktívna LOW),
  \item \textbf{Bzučiak}: GPIO15 (D8) -- zapojenie cez tranzistor a odpor; pin musí byť pri štarte v stave LOW.
\end{itemize}

\begin{table}[h!]
  \centering
  \caption{Zapojenie indikačných prvkov (ESP-12F)}
  \label{tab:feedback-wiring}
  \begin{tabular}{lll}
    \toprule
    Prvok & ESP-12F (ESP8266) & Poznámka \\
    \midrule
    LED (status) & GPIO2 (D4) & aktívna LOW, často vstavaná LED \\
    Bzučiak & GPIO15 (D8) & cez tranzistor; dodržať LOW pri boote \\
    \bottomrule
  \end{tabular}
\end{table}

\subsubsection{Schéma zapojenia a fotodokumentácia}
Na obrázku~\ref{fig:hw-prototype} je vyhradené miesto pre schému zapojenia alebo fotodokumentáciu prototypu. Obrázok bude doplnený podľa finálneho fyzického zapojenia.

\begin{figure}[h!]
  \centering
  \includegraphics[width=0.9\textwidth]{figures/circuit_image.png}
  \caption{Hardvérové zapojenie prototypu (doplní sa)}
  \label{fig:hw-prototype}
\end{figure}

\subsubsection{Firmvér zariadenia (\texttt{isic-project-hardware})}
Softvérová časť hardvérového zariadenia (firmvér pre ESP8266/ESP-12F) je implementovaná v samostatnom repozitári \texttt{isic-project-hardware}. Projekt je určený na kompiláciu cez \textit{PlatformIO} (Arduino framework) a používa štandard \textit{C++17} (viď \texttt{platformio.ini}).

Z pohľadu štruktúry kódu:
\begin{itemize}
  \item vstupný bod firmvéru je v súbore \texttt{src/main.cpp} (funkcie \texttt{setup()} a \texttt{loop()}),
  \item trieda \texttt{App} (\texttt{src/App.cpp}) inicializuje a koordinuje jednotlivé služby,
  \item služby sú implementované v \texttt{src/services/} a komunikujú nepriamo cez \texttt{EventBus} (publish/subscribe), čo znižuje väzby medzi modulmi.
\end{itemize}

Kľúčové služby firmvéru:
\begin{itemize}
  \item \texttt{Pn532Service} -- čítanie NFC kariet cez SPI a generovanie udalosti \textit{CardScanned},
  \item \texttt{WiFiService} a \texttt{MqttService} -- pripojenie do siete a odosielanie udalostí na broker,
  \item \texttt{AttendanceService} -- debouncing, prípadné dávkovanie udalostí a offline buffer,
  \item \texttt{ConfigService} -- načítanie/uloženie konfigurácie (napr.\ LittleFS) a zmena nastavení za behu,
  \item \texttt{OtaService} -- OTA aktualizácie firmvéru,
  \item \texttt{HealthService} -- diagnostika a zdravotný stav komponentov,
  \item \texttt{PowerService} -- riadenie úsporných režimov,
  \item \texttt{FeedbackService} -- LED/bzučiak pre spätnú väzbu používateľovi.
\end{itemize}

Konfigurácia zapojenia PN532 (SPI piny a IRQ) je v kóde reprezentovaná štruktúrou \texttt{Pn532Config} v súbore \texttt{include/common/Config.hpp}; tabuľka~\ref{tab:pn532-esp12f-wiring} zodpovedá predvolenému nastaveniu pre platformu ESP8266.

\subsubsection{Detailnejšia schéma firmvéru (vrstvy a služby)}
Prehľad firmvéru je rozdelený do vrstiev podobne ako v podnikových systémoch. Vrstvy sú oddelené tak, aby bolo možné jednoducho udržiavať kód, testovať jednotlivé časti a bezpečne rozširovať funkcionalitu bez zásahov do celého systému.

\begin{figure}[h!]
  \centering
  \begin{minipage}{\textwidth}
  {\scriptsize\begin{verbatim}
┌─────────────────────────────────────────────────────────────────────────────┐
│                              APPLICATION LAYER                              │
├─────────────────────────────────────────────────────────────────────────────┤
│  ┌───────────────────────────────────────────────────────────────────────┐  │
│  │                                App                                    │  │
│  │         Main coordinator • Service lifecycle • Task setup             │  │
│  └───────────────────────────────────┬───────────────────────────────────┘  │
├──────────────────────────────────────┼──────────────────────────────────────┤
│                              SERVICE LAYER                                  │
├──────────────────────────────────────┼──────────────────────────────────────┤
│                                      ▼                                      │
│  ┌───────────────────────────────────────────────────────────────────────┐  │
│  │                           EventBus                                    │  │
│  │           Signal/Slot pub/sub • Type-safe events • RAII connections   │  │
│  └───────────────────────────────────┬───────────────────────────────────┘  │
│          ┌───────────┬───────────┬───┴───┬───────────┬───────────┐          │
│          ▼           ▼           ▼       ▼           ▼           ▼          │
│  ┌─────────────┐ ┌─────────┐ ┌────────┐ ┌─────────┐ ┌─────────┐ ┌────────┐  │
│  │ConfigService│ │  WiFi   │ │  MQTT  │ │   OTA   │ │PN532    │ │Feedback│  │
│  │• LittleFS   │ │ Service │ │Service │ │ Service │ │Service  │ │Service │  │
│  │• JSON parse │ │• AP mode│ │• Queue │ │• Elegant│ │• SPI    │ │• LED   │  │
│  └─────────────┘ └─────────┘ └────────┘ └─────────┘ └─────────┘ └────────┘  │
│          │                       │           │           │                  │
│          ▼                       ▼           ▼           ▼                  │
│  ┌─────────────────────┐  ┌─────────────────────┐  ┌─────────────────────┐  │
│  │  AttendanceService  │  │    HealthService    │  │    PowerService     │  │
│  │  • Debounce/batch   │  │  • Component checks │  │  • Sleep modes      │  │
│  │  • Offline buffer   │  │  • MQTT reporting   │  │  • Signal-based     │  │
│  └─────────────────────┘  └─────────────────────┘  └─────────────────────┘  │
│                                                                             │
│                           ┌─────────────────────┐                           │
│                           │   TaskScheduler     │                           │
│                           │  • Cooperative      │                           │
│                           │  • Non-blocking     │                           │
│                           └─────────────────────┘                           │
├─────────────────────────────────────────────────────────────────────────────┤
│                              HARDWARE LAYER                                 │
├─────────────────────────────────────────────────────────────────────────────┤
│  ┌──────────┐  ┌──────────┐  ┌──────────┐  ┌──────────┐  ┌──────────┐       │
│  │   SPI    │  │   WiFi   │  │ LittleFS │  │   GPIO   │  │  Flash   │       │
│  │  (PN532) │  │ (MQTT)   │  │ (Config) │  │(LED/Buzz)│  │  (OTA)   │       │
│  └──────────┘  └──────────┘  └──────────┘  └──────────┘  └──────────┘       │
└─────────────────────────────────────────────────────────────────────────────┘
  \end{verbatim}}
  \end{minipage}
  \caption{Vrstvová schéma firmvéru zariadenia}
  \label{fig:firmware-layered}
\end{figure}

\subsubsection{Tok udalostí v rámci firmvéru}
Vstupná udalosť (sken karty) sa publikuje cez \textbf{EventBus} do \texttt{AttendanceService}, kde sa uplatní debounce a prípadné dávkovanie. Následne ju \texttt{MqttService} publikuje do brokeru. Konfigurácia sa spracúva cez \texttt{ConfigService} a stav zariadenia je pravidelne reportovaný službou \texttt{HealthService}. Táto architektúra umožňuje pridávať nové služby (napr.\ lokálne logovanie alebo ďalšie integračné výstupy) bez zásahu do existujúceho kódu.

\subsection{Backend: server a databáza}
Serverová časť je realizovaná ako serverová služba s \textit{REST API} a podporou prijímania udalostí z hardvérových zariadení cez \textit{MQTT}. V projekte sa používajú open-source technológie, najmä \textit{Python}, a taktiež kontajnerizácia pomocou \textit{Docker}, ktorá zjednodušuje nasadenie systému a zaručuje reprodukovateľnosť prostredia.

Backend zabezpečuje kľúčové funkcie:
\begin{itemize}
  \item prijímanie a spracovanie udalostí zo zariadení,
  \item prácu s databázou, v ktorej sú uložené údaje o študentoch, kurzoch, hodinách a záznamoch dochádzky,
  \item poskytovanie API pre frontend, teda pre webové rozhranie vyučujúceho/administrátora,
  \item tvorbu výstupov a podporu exportu/importu v formátoch CSV/Excel/XML pre ďalšie použitie alebo integráciu s inými systémami.
\end{itemize}

Serverová časť teda predstavuje centrálny prvok, ktorý spája čítacie zariadenia a používateľské webové rozhranie a zabezpečuje konzistentné ukladanie a sprístupnenie údajov.

\subsection{Frontend: webové rozhranie}
Používatelia pracujú so systémom prostredníctvom webového rozhrania. V projekte je navrhnutá webová aplikácia s moderným UI, implementovaná pomocou \textit{React}. Webová časť poskytuje prístup k údajom o dochádzke a nástroje na správu výučbových objektov.

Hlavné funkcie rozhrania:
\begin{itemize}
  \item prihlásenie do systému pre vyučujúcich a administrátorov,
  \item správa kurzov a študentov (prehľad, pridávanie/úprava, priraďovanie študentov ku kurzom),
  \item prehľad dochádzky podľa hodín alebo skupín,
  \item export výsledkov do požadovaných formátov na ďalšie spracovanie alebo prenos.
\end{itemize}


\subsection{Databáza - dátový model}

Na základe analýzy požiadaviek na systém evidencie dochádzky sme navrhli dátový model, ktorý pokrýva všetky základné potreby kompletného systému. Model pozostáva zo šiestich hlavných entít: User, Student, Subject, Room, Session a Attendance.

\subsubsection{Entita User}

Entita User reprezentuje používateľa systému -- vyučujúceho alebo administrátora:

\begin{itemize}
    \item \texttt{id} -- primárny kľúč, automaticky generovaný identifikátor záznamu
    \item \texttt{email} -- emailová adresa používateľa (UNIQUE), slúži ako prihlasovacie meno
    \item \texttt{password\_hash} -- zahašované heslo používateľa
    \item \texttt{role} -- rola používateľa v systéme (teacher, admin)
    \item \texttt{first\_name} -- krstné meno používateľa
    \item \texttt{last\_name} -- priezvisko používateľa
    \item \texttt{created\_at} -- časová značka vytvorenia záznamu
\end{itemize}

Rozlíšenie rolí umožňuje implementáciu rôznych úrovní prístupu: administrátori majú plný prístup k správe systému, zatiaľ čo vyučujúci majú prístup len k svojim predmetom a hodinám.

\subsubsection{Entita Student}

Entita Student reprezentuje študenta identifikovaného prostredníctvom karty ISIC:

\begin{itemize}
    \item \texttt{id} -- primárny kľúč, automaticky generovaný identifikátor záznamu
    \item \texttt{isic\_uid} -- jedinečný identifikátor čipu karty ISIC (UNIQUE, INDEX)
    \item \texttt{first\_name} -- krstné meno študenta
    \item \texttt{last\_name} -- priezvisko študenta
    \item \texttt{email} -- emailová adresa študenta
    \item \texttt{created\_at} -- časová značka vytvorenia záznamu
\end{itemize}

Atribút \texttt{isic\_uid} je indexovaný pre rýchle vyhľadávanie pri skenovaní karty. Osobné údaje môžu byť doplnené neskôr prostredníctvom administračného rozhrania alebo importom zo študentského informačného systému.

\subsubsection{Entita Subject}

Entita Subject reprezentuje vyučovaný predmet:

\begin{itemize}
    \item \texttt{id} -- primárny kľúč
    \item \texttt{code} -- kód predmetu (napr. ``TP1'', ``ZPRO'')
    \item \texttt{name} -- názov predmetu
    \item \texttt{created\_at} -- časová značka vytvorenia záznamu
\end{itemize}

\subsubsection{Entita Room}

Entita Room reprezentuje učebňu alebo miestnosť, kde prebieha výučba:

\begin{itemize}
    \item \texttt{id} -- primárny kľúč
    \item \texttt{name} -- názov alebo číslo miestnosti (napr. ``PK6-108'')
    \item \texttt{building} -- budova, v ktorej sa miestnosť nachádza
    \item \texttt{device\_id} -- identifikátor hardvérového zariadenia (čítačky) v miestnosti
\end{itemize}

\subsubsection{Entita Session}

Entita Session reprezentuje konkrétnu vyučovaciu hodinu:

\begin{itemize}
    \item \texttt{id} -- primárny kľúč
    \item \texttt{subject\_id} -- cudzí kľúč odkazujúci na predmet
    \item \texttt{room\_id} -- cudzí kľúč odkazujúci na miestnosť
    \item \texttt{teacher\_id} -- cudzí kľúč odkazujúci na vyučujúceho (User)
    \item \texttt{scheduled\_start} -- plánovaný začiatok hodiny
    \item \texttt{scheduled\_end} -- plánovaný koniec hodiny
    \item \texttt{status} -- stav hodiny (plánovaná, aktívna, ukončená, zrušená)
\end{itemize}

\subsubsection{Entita Attendance}

Entita Attendance reprezentuje záznam o účasti študenta na vyučovaní:

\begin{itemize}
    \item \texttt{id} -- primárny kľúč
    \item \texttt{student\_id} -- cudzí kľúč odkazujúci na študenta
    \item \texttt{session\_id} -- cudzí kľúč odkazujúci na vyučovaciu hodinu
    \item \texttt{scan\_timestamp} -- časová značka skenovania karty
    \item \texttt{status} -- stav účasti (prítomný, meškajúci, ospravedlnený, neprítomný)
    \item \texttt{created\_at} -- časová značka vytvorenia záznamu v databáze
\end{itemize}

Rozlíšenie medzi atribútmi \texttt{scan\_timestamp} a \texttt{created\_at} umožňuje korektné spracovanie situácií, kedy je záznam vytvorený s oneskorením (napríklad pri offline režime zariadenia).

\subsubsection{Vzťahy medzi entitami}

Obrázok \ref{fig:er-diagram} zobrazuje ER diagram navrhnutého dátového modelu so všetkými vzťahmi.

\begin{figure}[!ht]
    \centering
    \includegraphics[width=0.95\textwidth]{figures/er-diagram.png}
    \caption{ER diagram databázového modelu systému evidencie dochádzky}
    \label{fig:er-diagram}
\end{figure}

Medzi entitami existujú nasledujúce vzťahy typu 1:N (one-to-many):

\begin{itemize}
    \item \textbf{User -- Session}: Jeden vyučujúci môže viesť veľa vyučovacích hodín
    \item \textbf{Subject -- Session}: Jeden predmet môže mať veľa vyučovacích hodín
    \item \textbf{Room -- Session}: Jedna miestnosť môže hostiť veľa vyučovacích hodín
    \item \textbf{Student -- Attendance}: Jeden študent môže mať veľa záznamov o dochádzke
    \item \textbf{Session -- Attendance}: Jedna vyučovacia hodina môže mať veľa záznamov o dochádzke
\end{itemize}

Kombinácia cudzích kľúčov \texttt{student\_id} a \texttt{session\_id} v tabuľke Attendance je jedinečná (UNIQUE constraint), čo zabraňuje duplicitným záznamom o účasti toho istého študenta na tej istej hodine.

\subsubsection{Výhody navrhnutého modelu}

Navrhnutý dátový model má niekoľko výhod:

\begin{itemize}
    \item \textbf{Kompletnosť} -- model pokrýva všetky aspekty evidencie dochádzky vrátane správy používateľov, predmetov, miestností a vyučovacích hodín
    \item \textbf{Normalizácia} -- štruktúra je v tretej normálnej forme (3NF), čo eliminuje redundanciu údajov
    \item \textbf{Flexibilita} -- model umožňuje sledovanie dochádzky na úrovni jednotlivých hodín s možnosťou rozlíšenia stavu účasti
    \item \textbf{Auditovateľnosť} -- časové značky umožňujú sledovanie histórie záznamov a detekciu anomálií
    \item \textbf{Integrita} -- cudzí kľúče a jedinečné obmedzenia zabezpečujú konzistenciu údajov
\end{itemize}


\subsection{Architektonický návrh a interakcie komponentov}
Z hľadiska prevádzky je systém rozdelený na \textbf{zariadenie v učebni} (čítanie karty), \textbf{komunikačnú vrstvu} (Wi-Fi a MQTT), \textbf{serverovú logiku} (spracovanie udalostí a REST API), \textbf{databázu} (perzistencia) a \textbf{webové rozhranie} (správa a reporting). Pre prehľadnosť je základné prepojenie komponentov znázornené na obrázku~\ref{fig:attendance-architecture}.

\begin{figure}[h!]
  \centering
  {\setlength{\fboxsep}{6pt}%
  \begin{tabular}{c}
    \fbox{\parbox{0.9\textwidth}{\centering
      \textbf{Zariadenie v učebni} (ESP-12F + PN532 + LED/bzučiak, 3,3~V)\\
      čítanie UID karty, lokálna spätná väzba, vytvorenie udalosti \texttt{scan}
    }}\\[4pt]
    $\downarrow$\\[4pt]
    \fbox{\parbox{0.9\textwidth}{\centering
      \textbf{Wi-Fi + MQTT} (publish do témy, napr.\ \texttt{\{base\_topic\}/\{deviceId\}/attendance})
    }}\\[4pt]
    $\downarrow$\\[4pt]
    \fbox{\parbox{0.9\textwidth}{\centering
      \textbf{MQTT broker} (napr.\ Mosquitto)
    }}\\[4pt]
    $\downarrow$\\[4pt]
    \fbox{\parbox{0.9\textwidth}{\centering
      \textbf{Backend} (MQTT subscriber + REST API)\\
      spracovanie udalostí, REST API pre webové rozhranie
    }}\\[4pt]
    $\downarrow$\\[4pt]
    \fbox{\parbox{0.9\textwidth}{\centering
      \textbf{Databáza} (kurzy, rozvrh hodín, karty, záznamy dochádzky)
    }}\\
  \end{tabular}}
  \caption{Architektúra a prepojenie komponentov systému}
  \label{fig:attendance-architecture}
\end{figure}

Hardvérové zariadenie nevykonáva komplexnú aplikačnú logiku (napr.\ mapovanie karty na študenta alebo vyhodnotenie, ku ktorej hodine sken patrí). Jeho úlohou je \textbf{spoľahlivo zachytiť udalosť} a odoslať ju do komunikačnej vrstvy. Všetky rozhodnutia, validácie a zápisy do databázy sa vykonávajú na serveri, čo zjednodušuje správu viacerých zariadení a umožňuje centralizované zmeny pravidiel.

\section{Ako prebieha zaznamenanie prítomnosti}

Na obrázku~\ref{fig:attendance-sequence} je znázornený základný tok údajov
pri zaznamenaní prítomnosti od priloženia ISIC karty až po zobrazenie výsledkov
vo webovej aplikácii.

\begin{figure}[h!]
  \centering
  \includegraphics[width=\textwidth]{figures/attendance_sequence.png}
  \caption{Sekvenčný diagram procesu zaznamenania prítomnosti}
  \label{fig:attendance-sequence}
\end{figure}

\subsection{Krok 1: Skenovanie ISIC karty}
Študent príde na vyučovanie a priloží ISIC kartu k NFC čítačke v učebni. Zariadenie načíta identifikátor karty a vytvorí udalosť dochádzky, ktorá obsahuje základné technické údaje (identifikátor karty, identifikátor zariadenia, čas načítania a~pod.).

\subsection{Krok 2: Odoslanie udalosti na server}
Po načítaní karty zariadenie cez Wi-Fi odošle vytvorenú udalosť na server pomocou \textit{MQTT}. Tento prístup je vhodný aj pri väčšom počte učební a zariadení, pretože výmena správ je rýchla a rozšírenie o ďalšie zariadenia si nevyžaduje zásadné zmeny architektúry.

\subsection{Krok 3: Spracovanie na backende}
Backend prijme udalosť a vykoná základnú aplikačnú logiku:
\begin{itemize}
  \item overí, či je karta systému známa (t.\ j.\ či je priradená konkrétnemu študentovi),
  \item určí, ku ktorej hodine sa udalosť vzťahuje (napr.\ aktívna hodina pre danú učebňu alebo časové okno vyučovania),
  \item vytvorí alebo aktualizuje záznam v tabuľke dochádzky.
\end{itemize}

Následne sa udalosť stáva súčasťou štruktúrovaných údajov v databáze, s ktorými dokáže pracovať webové rozhranie.

\subsection{Krok 4: Zobrazenie a výstupy vo webovej aplikácii}
Vyučujúci alebo administrátor použije webovú aplikáciu na zobrazenie výsledkov: vyberie kurz a konkrétnu hodinu a zobrazí sa zoznam študentov s príslušnými záznamami prítomnosti. V prípade potreby je možné údaje exportovať (alebo importovať) v štandardných formátoch (CSV/Excel/XML).
