\chapter{Ako funguje ISIC Attendance System}

Táto kapitola stručne vysvetľuje princíp fungovania prototypu ISIC Attendance System z pohľadu architektúry a spracovania údajov. Cieľom je opísať, akým spôsobom systém prepája fyzickú identifikáciu študenta pomocou ISIC karty s digitálnym záznamom dochádzky, ktorý je následne dostupný vo webovej aplikácii.

\section{Hlavné komponenty systému}

Systém je navrhnutý ako modulárne riešenie pozostávajúce z troch častí: hardvérového zariadenia v učebni, serverovej časti a webového rozhrania. Jednotlivé komponenty spolu komunikujú tak, aby bolo možné spoľahlivo zachytiť udalosť priloženia karty, preniesť ju cez sieť, spracovať na serveri a uložiť do databázy.

\subsection{Hardware: zariadenie v učebni}
Na evidenciu prítomnosti sa v učebni, kde je potrebné zaznamenávať dochádzku, umiestni kompaktné zariadenie s RFID/NFC čítačkou. V rámci prototypu je hardvérová časť postavená na mikrokontroléri \textit{ESP8266 / ESP12-F} a NFC čítačke \textit{PN532}. Zariadenie sa pripája do lokálnej Wi-Fi siete a odosiela udalosti na server. Na prenos správ sa využíva protokol \textit{MQTT}, ktorý je vhodný pre scenáre s väčším počtom zariadení a častými udalosťami, keďže umožňuje ľahkú a rýchlu výmenu dát.

\subsection{Backend: server a databáza}
Serverová časť je realizovaná ako serverová služba s \textit{REST API} a podporou prijímania udalostí z hardvérových zariadení cez \textit{MQTT}. V projekte sa používajú open-source technológie, najmä \textit{Python}, a taktiež kontajnerizácia pomocou \textit{Docker}, ktorá zjednodušuje nasadenie systému a zaručuje reprodukovateľnosť prostredia.

Backend zabezpečuje kľúčové funkcie:
\begin{itemize}
  \item prijímanie a spracovanie udalostí zo zariadení,
  \item prácu s databázou, v ktorej sú uložené údaje o študentoch, kurzoch, hodinách a záznamoch dochádzky,
  \item poskytovanie API pre frontend, teda pre webové rozhranie vyučujúceho/administrátora,
  \item tvorbu výstupov a podporu exportu/importu v formátoch CSV/Excel/XML pre ďalšie použitie alebo integráciu s inými systémami.
\end{itemize}

Serverová časť teda predstavuje centrálny prvok, ktorý spája čítacie zariadenia a používateľské webové rozhranie a zabezpečuje konzistentné ukladanie a sprístupnenie údajov.

\subsection{Frontend: webové rozhranie}
Používatelia pracujú so systémom prostredníctvom webového rozhrania. V projekte je navrhnutá webová aplikácia s moderným UI, implementovaná pomocou \textit{React}. Webová časť poskytuje prístup k údajom o dochádzke a nástroje na správu výučbových objektov.

Hlavné funkcie rozhrania:
\begin{itemize}
  \item prihlásenie do systému pre vyučujúcich a administrátorov,
  \item správa kurzov a študentov (prehľad, pridávanie/úprava, priraďovanie študentov ku kurzom),
  \item prehľad dochádzky podľa hodín alebo skupín,
  \item export výsledkov do požadovaných formátov na ďalšie spracovanie alebo prenos.
\end{itemize}

\section{Ako prebieha zaznamenanie prítomnosti}

Na obrázku~\ref{fig:attendance-sequence} je znázornený základný tok údajov
pri zaznamenaní prítomnosti od priloženia ISIC karty až po zobrazenie výsledkov
vo webovej aplikácii.

\begin{figure}[h!]
  \centering
  \includegraphics[width=\textwidth]{figures/attendance_sequence.png}
  \caption{Sekvenčný diagram procesu zaznamenania prítomnosti}
  \label{fig:attendance-sequence}
\end{figure}

\subsection{Krok 1: Skenovanie ISIC karty}
Študent príde na vyučovanie a priloží ISIC kartu k NFC čítačke v učebni. Zariadenie načíta identifikátor karty a vytvorí udalosť dochádzky, ktorá obsahuje základné technické údaje (identifikátor karty, identifikátor zariadenia, čas načítania a~pod.).

\subsection{Krok 2: Odoslanie udalosti na server}
Po načítaní karty zariadenie cez Wi-Fi odošle vytvorenú udalosť na server pomocou \textit{MQTT}. Tento prístup je vhodný aj pri väčšom počte učební a zariadení, pretože výmena správ je rýchla a rozšírenie o ďalšie zariadenia si nevyžaduje zásadné zmeny architektúry.

\subsection{Krok 3: Spracovanie na backende}
Backend prijme udalosť a vykoná základnú aplikačnú logiku:
\begin{itemize}
  \item overí, či je karta systému známa (t.\ j.\ či je priradená konkrétnemu študentovi),
  \item určí, ku ktorej hodine sa udalosť vzťahuje (napr.\ aktívna hodina pre danú učebňu alebo časové okno vyučovania),
  \item vytvorí alebo aktualizuje záznam v tabuľke dochádzky.
\end{itemize}

Následne sa udalosť stáva súčasťou štruktúrovaných údajov v databáze, s ktorými dokáže pracovať webové rozhranie.

\subsection{Krok 4: Zobrazenie a výstupy vo webovej aplikácii}
Vyučujúci alebo administrátor použije webovú aplikáciu na zobrazenie výsledkov: vyberie kurz a konkrétnu hodinu a zobrazí sa zoznam študentov s príslušnými záznamami prítomnosti. V prípade potreby je možné údaje exportovať (alebo importovať) v štandardných formátoch (CSV/Excel/XML).
