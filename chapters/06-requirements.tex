\chapter{Špecifikácia požiadaviek a use-case scenáre}
\label{chap:requirements-usecases}

Táto kapitola špecifikuje funkčné a nefunkčné požiadavky systému \textit{ISIC Attendance}. Systém slúži na evidenciu účasti študentov na cvičeniach/prednáškach pomocou priloženia ISIC karty k čítačke.

\section{Funkčné požiadavky}
\label{sec:functional-requirements}

Funkčné požiadavky definujú, čo má systém robiť a aké služby poskytuje jednotlivým používateľom.

\begin{enumerate}
    \item \textbf{Načítanie identifikátora z ISIC karty.}  
    Systém musí vedieť načítať unikátny identifikátor z ISIC karty (NFC/RFID) pomocou hardvéru a tento identifikátor ďalej spracovať.

    \item \textbf{Identifikácia študenta.}  
    Po načítaní identifikátora systém vyhľadá príslušného študenta v evidencii. V prípade, že identifikátor nie je priradený žiadnemu študentovi, systém túto situáciu vyhodnotí ako neúspešnú.

    \item \textbf{Zaznamenanie účasti na konkrétnej hodine.}  
    Po úspešnej identifikácii študenta systém vytvorí záznam o účasti viazaný na konkrétnu hodinu (dátum, čas, predmet/skupina alebo iný identifikátor hodiny).

    \item \textbf{Kontrola duplicít.}  
    Systém pred vytvorením záznamu overí, či už študent nie je na danej hodine evidovaný ako prítomný, aby nedochádzalo k duplicite.

    \item \textbf{Overenie aktívnej hodiny (okna registrácie).}  
    Systém musí vedieť rozlíšiť, či je v danom čase aktívna hodina. Ak registrácia nie je povolená, systém nezapíše účasť.

    \item \textbf{Spätná väzba pre študenta na mieste.}  
    Po priložení karty systém poskytne okamžitú informáciu o výsledku operácie (úspech/neúspech). Forma spätnej väzby --- LED.

    \item \textbf{Prehľad účasti pre vyučujúceho.}  
    Vyučujúci má mať k dispozícii prehľad študentov, ktorí sa na hodinu zaregistrovali, a základnú štatistiku.

    \item \textbf{Uloženie dát do trvalého úložiska.}  
    Záznamy o účasti musia byť ukladané tak, aby boli dostupné aj po reštarte systému.

    \item \textbf{Export výsledkov.}  
    Systém umožní exportovať výsledky (napr. zoznam prítomných) do formátu vhodného na ďalšie spracovanie.

    \end{enumerate}

\section{Nefunkčné požiadavky}
\label{sec:nonfunctional-requirements}

Nefunkčné požiadavky definujú kvalitatívne vlastnosti systému, teda ako dobre má systém fungovať.

\begin{enumerate}
    \item \textbf{Spoľahlivosť a odolnosť voči chybám.}  
    Systém má korektne reagovať na typické problémové situácie, ako je opakované priloženie karty, neúspešné načítanie identifikátora, alebo dočasná nedostupnosť backendu (ak je použité sieťové riešenie). V prípade chyby musí byť stav jednoznačný a zrozumiteľný.

    \item \textbf{Integrita dát.}  
    Systém má zabezpečiť konzistentnosť údajov: pre jedného študenta a jednu hodinu sa má evidovať najviac jeden záznam účasti.

    \item \textbf{Bezpečnosť a prístupové práva.}  
    Údaje o účasti predstavujú osobné údaje. Prístup k prehľadom a exportom má byť obmedzený na oprávnené roly (vyučujúci, administrátor). Systém nemá zverejňovať osobné údaje nepovolaným osobám.

    \item \textbf{Udržiavateľnosť.}  
    Softvérová časť má byť navrhnutá tak, aby bolo možné systém rozširovať bez zásahov do celej architektúry.

    \item \textbf{Logovanie a diagnostika.}  
    Systém má uchovávať záznamy o dôležitých udalostiach (načítanie karty, úspešný/neúspešný zápis, dôvody zamietnutia), čo umožní jednoduchšie odhalenie problémov pri prevádzke.
\end{enumerate}
