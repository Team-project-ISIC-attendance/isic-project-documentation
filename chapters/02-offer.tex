\chapter{Ponuka na vypracovanie tímového projektu}

Táto kapitola predstavuje ponuku tímu na vypracovanie tímového projektu z predmetu
\textit{Tímový projekt}. Ponuka slúži ako záväzný dokument, ktorého cieľom je preukázať
odbornú spôsobilosť tímu riešiť zvolenú tému, jasne definovať motiváciu tímu a načrtnúť
navrhované technické riešenie projektu.

Tím navrhuje realizáciu projektu s názvom \textbf{Systém na evidenciu účasti študentov na hodinách},
ktorého cieľom je modernizácia a automatizácia evidencie dochádzky v akademickom prostredí
prostredníctvom kombinácie hardvérového riešenia a webovej aplikácie.

\section{Predstavenie tímu}

Tím pozostáva z piatich študentov, ktorí spoločne disponujú znalosťami a skúsenosťami
v oblasti softvérového inžinierstva, webových technológií, práce s hardvérom a návrhu
používateľských rozhraní. Všetci členovia tímu majú skúsenosti s univerzitnými projektmi,
ako aj s praktickým vývojom softvéru alebo hardvéru.

\begin{itemize}
    \item \textbf{Andrian-Maksym Balaichuk} – softvérový inžinier so skúsenosťami z komerčného prostredia,
    zameraný na programovanie v jazyku C++ a priamu komunikáciu s hardvérom.
    \item \textbf{Danylo Zahorulko} – backend vývojár so silným zameraním na jazyk Python,
    návrh databázových štruktúr a implementáciu API rozhraní.
    \item \textbf{Vladyslav Panik} – UX/UI dizajnér so skúsenosťami s návrhom používateľsky orientovaných
    riešení a prototypovaním webových aplikácií.
    \item \textbf{Vitalii Romaniuk} – softvérový vývojár so znalosťami nízkoúrovňového programovania,
    mikrokontrolérov a komunikácie so senzormi.
    \item \textbf{Yurii Soma} – frontend vývojár so skúsenosťami s technológiami React a TypeScript,
    zameraný na vývoj moderných, výkonných a prístupných používateľských rozhraní.
\end{itemize}

Všetci členovia tímu počas bakalárskeho štúdia nadobudli skúsenosti s prácou na projektoch
zameraných na mikrokontroléry (Arduino, ESP32, ESP8266) a webové aplikácie, čo vytvára
pevný základ pre úspešnú realizáciu navrhovaného projektu.

\section{Motivácia tímu}

Hlavnou motiváciou tímu je potreba modernizácie procesu evidencie dochádzky na vysokých školách.
V súčasnosti je dochádzka často zaznamenávaná manuálne, čo je časovo náročné, náchylné na chyby
a môže viesť k nepresnostiam alebo zneužitiu.

Tím má záujem:
\begin{itemize}
    \item navrhnúť transparentný a spoľahlivý systém evidencie účasti,
    \item znížiť administratívnu záťaž pedagógov,
    \item aplikovať moderné technológie v reálnom akademickom prostredí.
\end{itemize}

Projekt zároveň poskytuje tímu príležitosť získať praktické skúsenosti s návrhom komplexného
systému, ktorý kombinuje hardvérovú a softvérovú časť, čo zodpovedá reálnym požiadavkám
praxe v oblasti IT a IoT riešení.

\subsection{Čo môže tím poskytnúť}

Tím navrhuje vytvorenie prototypu systému, ktorý bude pozostávať z dvoch hlavných častí:
hardvérového zariadenia a webovej aplikácie.

\textbf{Hardvérová časť}:
\begin{itemize}
    \item návrh a implementácia RFID zariadenia založeného na mikrokontroléri (ESP8266/ESP12-F),
    \item komunikácia zariadenia so serverom prostredníctvom Wi-Fi (mqtt),
    \item jednoduchá konfigurácia a nasadenie v učebniach.
\end{itemize}

\textbf{Softvérová časť}:
\begin{itemize}
    \item webová aplikácia pre správu kurzov a evidenciu dochádzky,
    \item prihlasovanie pre učiteľov a administrátorov,
    \item databázový model pre študentov, predmety a záznamy dochádzky,
    \item prehľadné a moderné používateľské rozhranie.
\end{itemize}

\section{Predpokladané zdroje}

Na realizáciu projektu tím predpokladá využitie nasledovných zdrojov:

\begin{itemize}
    \item \textbf{Časové zdroje}: pravidelné týždenné stretnutia tímu (1–2 hodiny) a konzultácie s vedúcim projektu,
    \item \textbf{Hardvér}: jedno RFID zariadenie na testovanie a demonštráciu prototypu,
    \item \textbf{Softvér}: open-source technológie (React, Python, Docker, Nginx),
    \item \textbf{Serverové zdroje}: univerzitná infraštruktúra alebo VPS pre nasadenie aplikácie.
\end{itemize}

Predpokladané finančné náklady na hardvérovú časť projektu sú približne 15–20 € na jednu
kompletnú zostavu, čo je z hľadiska prototypu považované za prijateľné.

\section{Konštruktívne návrhy}

Tím navrhuje zavedenie čiastkových míľnikov počas realizácie projektu. Tento prístup umožní:
\begin{itemize}
    \item priebežné hodnotenie dosiahnutých výsledkov,
    \item včasnú spätnú väzbu od vedúceho projektu,
    \item lepšie plánovanie práce a rozdelenie úloh v tíme.
\end{itemize}

\section{Záver ponuky}

Predložená ponuka preukazuje odbornú pripravenosť tímu riešiť zvolenú tému,
jasne definuje ciele projektu a navrhované riešenie.
Tím je presvedčený, že navrhovaný systém má praktický prínos pre akademické prostredie
a jeho realizácia prispeje k splneniu vzdelávacích cieľov predmetu Tímový projekt.



